\documentclass[11pt]{article}
\usepackage[tmargin=1.25in,lmargin=1in,rmargin=1in,bmargin=1in,paper=letterpaper]{geometry}
\usepackage{amsmath,amssymb}
\usepackage{multirow,color}
\usepackage{fancyhdr,ifthen,lastpage}
% \usepackage[adobe-utopia]{mathdesign}
% \usepackage[T1]{fontenc}
\usepackage{palatino}
\usepackage{mathtools}					% Necessary for \coloneqq
\usepackage[x11names]{xcolor}			%	  "		 "	\color{...}, \textcolor{...}{...}, \colorbox{...}{...}
%\definecolor{Color Name}{HTML}{00F9DE}	%	  "		 "	defining new colors not available via x11names (for more info: https://mirrors.rit.edu/CTAN/macros/latex/contrib/xcolor/xcolor.pdf)
\usepackage[shortlabels]{enumitem}		%	  "		 "	changing the 'enumerate' list format
\newcommand{\cyanit}[1] {\textit{\textcolor{Cyan4}{#1}}}
\newcommand{\numlessfn}[1] {\begingroup	%	  "		 "	grouping and must be ended w/ \endgroup
\renewcommand{\thefootnote}{}			%	  "		 "	changing the numbering style of footnotes in the main document text
{\footnote{#1}}
\addtocounter{footnote}{-1}				% \addtocounter increments the counter (i.e. footnote) by the amount specified by the value argument (i.e. -1)
\endgroup}
\usepackage{pifont}						% More styles for bullets
%--------------------------------------------------------------------
\iffalse
\begin{center}
\includegraphics[width = \textwidth]{}
\end{center}
\fi
%--------------------------------------------------------------------
\begin{document}
\pagestyle{fancy}
% --------------------------------------------------------------------
%% HEADER %%
\lhead{SURF 2023}
\rhead{Camilo Morales}
\chead{\bf Abstract}

%% FOOTER %%
\lfoot{} 
\rfoot{}
%\cfoot{\ifthenelse{\equal{\thepage}{\pageref{lastpage}}}{}{\footnotesize{{\color{red}(Continued on next page)}}}}
% --------------------------------------------------------------------
	The Kochen-Specker Theorem reduces Bell’s argument on the noncontextuality of observables in quantum mechanics to a proof using a set of 117-three dimensional vectors in Hilbert space with no Kochen-Specker (KS) coloring. Since then, several KS uncolorable vector sets have been constructed, including a set of vectors whose rank-1 projection matrices have entries in the rational subring $\mathbb{Z}[1/462]$ found by Cortez and Reyes. The vectors that were generated correspond to the set $\mathcal{S}(N)$, for $N = 462$, which is defined as a subset of $\mathbb{Z}^{3}$ that contains vectors whose norm squared divides a power of $N$. However, the iterative processes used to generate the set of vectors $\mathcal{S}(462)$ becomes computationally expensive when trying to prove whether sets of vectors such as $\mathcal{S}(6)$ are KS uncolorable. To reduce the runtime of such procedures, we develop an algorithm that replaces some iterative processes with Python functions with at most $O(n)$ complexity. Furthermore, we leverage methods that solve Diophantine equations of the form $i^{2} + j^{2} + k^{2} = N$ to more efficiently construct the necessary orthogonal basis vectors. By implementing this algorithm, we are able to identify a contradiction in our proof earlier on and determine whether a finite set of integer vectors is KS colorable.
% --------------------------------------------------------------------
\end{document}